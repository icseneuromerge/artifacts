\section{Downstream Tasks}
\label{sec:downstream_tasks}

In the previous section we focus on data flow analysis as a benchmark for the
capabilities of machine learning for compiler analysis. For the analyses
considered, non-ML techniques achieve perfect scores. In this section we apply
\programl to two downstream data flow tasks for which non-ML techniques fail:
predicting heterogeneous compute device mappings and algorithm classification.
In both domains \programl outperforms prior graph-based and sequence-based
representations, reducing test error by $1.20\times$ and $1.35\times$,
respectively. Finally, we ablate every component of our representation and
summarize the contribution of each.

\subsection{Heterogeneous Device Mapping}

We apply our methodology to the challenging domain of heterogeneous compute
device mapping. Given an OpenCL kernel and a choice of two devices to run it on
(CPU or GPU), the task is to predict the device which will provide the best
performance. This problem has received significant prior attention, with
previous approaches using both hand-engineered features~\citep{Grewe2013} and
sequential models~\citep{Ben-nun2018,Cummins2017b}. We use the \textsc{OpenCL
Devmap} dataset~\citep{Cummins2017b}, which provides 680 labeled CPU/GPU
instances derived from 256 OpenCL kernels sourced from seven benchmark suites on
two combinations of CPU/GPU hardware, AMD and NVIDIA. \emph{cf.\ Appendix~D.1.
in supplementary materials for details.}

The performance of \programl and baseline models is shown in
Table~\ref{figure:devmap_results}. As can be seen, \programl outperforms prior
works. We set new state-of-the-art F$_1$ scores of 0.88 and 0.80.

\begin{table}
  \centering%
  \centering
\vspace{-.5em}
\footnotesize
\begin{tabular}{l r | r }
	& AMD & NVIDIA \\
	& Error [\%] & Error [\%] \\
	\toprule
	Static Mapping                 & 41.2 & 43.1\\
	DeepTune   & 28.1 & 39.0\\
	DeepTune$_{\text{IR}}$         & 26.2 & 31.6\\
	inst2vec    & 19.7 & 21.5\\
	\programl                       & \textbf{13.4} & 	\textbf{20.0}\\
\end{tabular}

  \caption{%
    Predicting heterogeneous compute device mapping.
  }%
  \label{figure:devmap_results} %
\end{table}

\begin{table}
  \centering
  \vspace{-.5em}
\centering
\footnotesize
\begin{subfigure}{.98\linewidth}
  \centering
  \begin{tabular}{l r r}
    & Error [\%] & Relative [\%]\\
    \toprule
    TBCNN & 6.00 & +77.5\\
    NCC & 5.17 & +53.0\\
    XFG w.\ inst2vec vocab & 4.56 & +34.9\\
    XFG & 4.29 & +26.9\\
    \programl & \textbf{3.38} & -\\
    &&\\
    &&\\
  \end{tabular}
  \vspace{-2.5em}
  \caption{Comparison to state-of-the-art.}
  \label{tab:classify_soa}
\end{subfigure}\\*
\vspace{.5em}
\begin{subfigure}{.98\linewidth}
  \centering
  \begin{tabular}{l r r}
    & Error [\%] & Relative [\%]\\
    \toprule
    No vocab & 3.70 & +9.5\\
    inst2vec vocab & 3.78 & +11.8\\
    No control edges & 3.88 & +14.8\\
    No data edges & 7.76 & +129.6\\
    No call edges & 3.88 & +14.8\\
    No backward edges & 4.16 & +23.1\\
    No edge positions & 3.43 & +1.5\\
  \end{tabular}
  \caption{\programl ablations.}
  \label{tab:classify_ablations}
\end{subfigure}

  \caption{%
    Algorithm classification comparison to state-of-the-art, and ablations.
  }
  \label{tab:classify}
\end{table}

\subsection{Algorithm Classification}

We apply our approach to the task of classifying algorithms from unlabeled
implementations. We use the~\citet{Mou2016} dataset. It contains implementations
of 104 different algorithms that were submitted to a judge system. All samples
were written by students in higher education. There are around 500 samples per
algorithm. We compile them with different combinations of optimization flags to
generate a dataset of overall 240k samples, as in~\citet{Ben-nun2018}.
Approximately 10,000 files are held out each as development and test sets.
\emph{cf.\ Appendix~D.2. for details}.

Table~\ref{tab:classify_soa} compares the test error of our method against prior
works, where we set a new state-of-the-art.

\paragraph{Ablation Studies} We ablate the \programl representation in
Table~\ref{tab:classify_ablations}. Every component of our representation
contributes positively to performance. We note that structure alone (\emph{No
vocab}) is sufficient to outperform prior work, suggesting that algorithm
classification is a problem that lends itself especially well to judging the
power of the representation structure, since most algorithms are well-defined
independent of implementation details, such as data types. However, the choice
of vocabulary is important. Replacing the \programl vocabulary with that of a
prior approach (\emph{inst2vec vocab}~\cite{Ben-nun2018}) degrades performance.
The greatest contribution to the performance of \programl on this task is data
flow edges. Backward edges~\cite{Li2015a}, which are required for reasoning
about backward data flow analyses, provide the second greatest contribution.
These results highlight the importance of data flow analysis for improving
program reasoning through machine learning.
